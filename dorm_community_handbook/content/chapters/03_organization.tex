\chapter{How we organize ourselves} \label{chap:organization}
Now that you know us and what life is like around here, let's look at how we organize ourselves.

\section{Keeping the kitchen clean}
Ah, yes, the kitchen. Keeping order with 15 students sharing one room is definitely a challenge.

But rest assured, we will do our best to keep the chaos at bay. Our organization's main objective has to be keeping the kitchen clean continuously and not relying on frequent cleanups. I can assure you that there is no way to maintain a cleanup schedule that would keep up with the paste; the kitchen will drown in misery.

That's why the first rule for the kitchen is: Ignore the official cleaning schedule; it's nonsense. Or would you like to clean up half a year's worth of built-up mess that no one cared about before and schedule your work and free time activities around cleaning duty? Told you it was nonsense.

Instead, we ask you to take one simple rule to heart: \textbf{Please clean up the mess you made yourself, and make sure to leave the kitchen \textit{at least} as clean as you found it.}

I want to emphasize the \quotes{at least} part in the above statement because the kitchen has gone through some unhealthy phases under the wrong mindset. Imagine the following situation: You go to the kitchen and find the stove a little dirty, so obviously, someone before you did a bad job at cleaning up. If you take care of it before or after using the stove, the situation is cleared within five minutes. But if you use the stove yourself and refuse to clean up afterward, because \quotes{you don't want to do someone else job}, you are not just leaving the problem to someone else, you made it worse. Because no matter how careful you are, some sprinkles probably snuck out of the pot. And even if not, the heat will have led to the previous mess drying and being harder to clean. The following person would have to commit even more to cleaning up to clear the situation. That's why this mindset spreads incredibly quickly, and in the end, everyone is angry at each other and unhappy with the situation.

So please, be forgiving with your room neighbors. I'm sure they will return the favor and have your back when you have a long day and miss a spot yourself. If you feel unhappy about something, please communicate and not build up anger. We all use the kitchen differently, and different cultural backgrounds can lead to other priorities. But so far, everyone I have met in this community has been willing to work towards betterment once an issue has been addressed.

Some final words to conclude this section: Even if some inhabitants don't care about the kitchen at all, as long as a majority is working together, the situation is manageable. So, let's join forces, shall we?

\subsection{Kitchen responsibilities}

Unfortunately, the kitchen does not only consist of flat surfaces that can be wiped clean after every use, and the job is done. There are some bigger jobs that need to be taken care of once in a while. Frequency and effort are very dependent on the kind of job and how much the object in question is used.

We devised a responsibility system to distribute the workload as fairly as possible. Let's go through the basics of this system, using the oven as an example:
\begin{enumerate}
    \item Everyone who uses the oven is still obligated to clean up the immediate mess they have made. For example, that includes rinsing the oven tray and getting rid of bigger splashes or drips from boiling food.
    \item If you don't clean up after yourself properly, the responsible person is not only allowed but even encouraged to contact you and demand that you take care of that issue.
    \item In turn, the responsible person will take care of a more thorough oven cleaning once in a while, e.g., getting rid of small splashes and the layer of fat that will build up over time on all oven surfaces from the food vapors.
    \item If someone finds the oven unacceptable, they can turn to the responsible person listed below.
    \item If you leave for longer periods (let's say, more than 4 days?), you should organize a stand-in that will take care of your responsibilities until you are back. Of course, you shouldn't leave the big tasks to them ;)
\end{enumerate}

If you read this the first time, you might be thinking \quotes{This just sounds like a blaming system!}, and I agree that wrongly executed this could lead to firing blame at each other. But the intention is quite the contrary. \quotes{Keeping the kitchen clean} can be daunting because there is a lot to do, so where do you start? The list below is supposed to show you that you are not alone in this effort and to break down this big task into small, manageable chunks that we can distribute and limit the workload of single people. I hope you can see the vision and maybe take responsibility for yourself!

The current responsibility distribution is in table \ref{tab:kitchen-responsibilities}.

\begin{table}[htp]
    \centering
    \begin{tabular}{ccc}
        \cline{1-1}
        \rowcolor[HTML]{F89646} 
        Responsibility                    & Person                & Room Number  \\
        Couch \& Speaker Area             & Robin Epple           & U 13         \\ \hline
        Dishes \& Washing Utilities       & Silvy Kurzendorfer    & U 17         \\ \hline
        Floor \& Window Sill              & Robin Epple           & U 13         \\ \hline
        Shelves                           &                       &              \\ \hline
        Freezer                           & Luke Caputo G.        & U 09         \\ \hline
        Fridge                            & Luke Caputo G.        & U 09         \\ \hline
        Microwave                         &                       &              \\ \hline
        Oven                              & Danny Löser           & U 08         \\ \hline
        Cooking Surfaces                  & Michael Stengel       & U 06         \\ \hline
        Pans, Pots, etc.                  &                       &              \\ \hline
        Table \& Chairs                   &                       &              \\ \hline
        Toaster, Kettle \& Sandwich Maker & Silvy Kurzendorfer    & U 17         \\ \hline
        Garbage \& Area around            & Luke C. and Bogdan T. & U 09 \& U 16 \\ \hline
        Sink \& Wall behind               &                       &              \\ \hline
    \end{tabular}
    \caption{The current kitchen-responsibility distribution.}
    \label{tab:kitchen-responsibilities}
\end{table}

\section{Keeping track of finances}
I hate to break it to you, but living costs money. Wouldn't it be awful to shift around cents every day for groceries, a trip to the cinema, etc.?

At least we felt like it was, so we decided to do things differently and open a group in the app \quotes{Splitwise}. If someone pays the expenses for multiple people, we don't immediately pay each other out. Instead, the total costs are entered in the app (mostly with even distribution among the participants, but individual amounts are also possible).

Ideally, expenses level out over time. But even if they don't, we can at least pay out debts at less frequent intervals.

If you want to participate in our cooking or other activities, we kindly ask you to download the app from the \mbox{\href{https://play.google.com/store/apps/details?id=com.Splitwise.SplitwiseMobile&hl=de&pli=1}{$\xrightarrow{}$ Play Store (Android)}} or \mbox{\href{https://apps.apple.com/de/app/splitwise/id458023433}{$\xrightarrow{}$ App Store (iOS)}}. You can find the invitation link to our group as a QR code in the kitchen.

\section{Where your groceries go}
The kitchen is divided into public and private storage areas. The area under the stove and the two unlabelled cupboard sections on the far left over the drying area are for public utilities, and everyone is free to use dishes.

Every room has a labeled cupboard section for private dishes and groceries that don't need cooling, either above the working area or in the back next to the couch. These are not very big, though, so if your items exceed the space in the kitchen, you either have to keep some in your room (which is what most of us do) or make individual arrangements with the other inhabitants to share spaces. 

We have two fridges next to the door for groceries that need cooling. Every room has one section. To find yours, start counting from the top in the left fridge with room two (yes, there is no room one on our floor) and count until you find your room number. The bottom tray is included in the counting and continued in the right fridge accordingly.

The compartments in the door are less organized, so just be sensible and don't use all of them. It is usually not an issue to find a spot for your items.

{\color{darkred} TODO Picture}

On the other side of the door, you find the freezer, which follows a similar approach. Since we only have one, each compartment is shared by two rooms. So start with (2/3) and count up from that.

{\color{darkred} TODO Picture}

As always, if you need more space, talk to us. There will be a solution. Please monitor your items and don't start experiments on growing new lifeforms in our fridge.