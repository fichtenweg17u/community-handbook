\chapter{I just arrived, what now?}

So you have just wandered into your new home for the first time and try to figure out what organizational steps you need to take to start your life here? Don't worry, we've all been there. And while the following list may not be comprehensive for you, this \quotes{Quickstart} can hopefully give you a starting point.

\section{I got keys, what are they for?}

You should have one key with teeth only on one side. That one is for the buildings main door and your room at the same time. Your room does not lock behind you automatically, but the front door will (except if a little switch is toggled), so don't forget them!

The other key with teeth on both sides is for your mail. You can find the mail room in the center of the WHO right next to the club \quotes{Kuckuck}. It is open during day hours and you should regularily check your mail. Have a look at section \ref{sec:qna-mail} for details.

All neighbors on one floor share one kitchen. As you can read in some later chapters, we decided to make the kitchen into a living space where we can meet and spend some time together. Just come and say hi whenever you like :). The washing machine is also on our floor, in the room at the bottom of the staircase, but is shared across the entire building. For details have a look at chapter \ref{sec:qna-washing-machine}.

\section{How do I register my residence?}

In Germany you are obligated to register in the city you are residing in, even if it is just for a few months. When you got the keys from the caretaker, you should have also received a letter. This \quotes{Wohnungsgeberbestätigung} is a confirmation, that you are renting a room here in WHO. With this letter you need to register at the \quotes{Bürgeramt}. As with every government-institution, it is of no use going there without an appointment, and the next free appointment might be a few weeks from now. But don't worry: You are obligated to register within the first couple of weeks of getting here, but it is enough if you schedule the appointment within that timespan, noone can blame you for having to wait for a free timeslot.

You can schedule an appointment \href{https://www.tuebingen.de/verwaltung/onlinedienste#/anmeldung_wohnsitz}{$\rightarrow{}$ here}. If you don't speak german and need help with the site, ask in chat, we are happy to help!

You can register your room as a secondary residence, but Tübingen will ask a yearly secondary-residence-tax of you if you do so. At the time of writing, this tax is ten percent of your yearly rent. You can find the details \href{https://www.tuebingen.de/verwaltung/uploads/zweitwohnungsteuersatzung.pdf}{$\rightarrow{}$ here}, though be aware that the linked PDF might not stay up to date!

\section{What the hell is the \quotes{Rundfunkbeitrag}?}

In germany we have a special tax, that finances public media like radio, public TV channels and more. It is payed per household, it does not matter if you plan on using it or not, and it is also obligatory when you are not a german citicen and only here for studying. There are special rules for flat-sharing communities, but unfortunately our dorm does not qualify for these, because we all have independently locked rooms. So yes, you need to pay the \quotes{Rundfunkbeitrag}. They will send you a letter eventually after they receive notice of your registration in Tübingen, but if you want to get it done early, you can register on \href{https://www.rundfunkbeitrag.de/buergerinnen_und_buerger/formulare/anmelden/index_ger.html}{$\rightarrow{}$ this site}. You will not safe any money, no matter when you register, they know exactly in what month you registered. As mentioned before, if you need help with the german site, just let us know.

\section{What about Uni?}

Since every study program is organized slightly different, we can't give you some globally applicable advice on how to register for courses etc. You will receive an account at some point and you will need to engage with the very user friendly tool (*not*) \href{https://alma.uni-tuebingen.de}{$\rightarrow{}$ Alma} sooner or later. For the specifics just talk to us and/or look out for introduction events for newbies at the Uni Tübingen. There are some formal ones like an introduction to alma, some events specifically for foreign students and lots of events individually organized by your \quotes{Fachschaft} (student body of your faculty). So keep an eye out for those!

One thing we can say for sure, is that you will need your student ID card as payment method for the washing machine. If you have not recieved it yet or haven't had a chance to top it up with money, contact us and we can help you out :)

\section{What recurring obligations do I need to be aware of?}

For living in WHO and studying in Tübingen you need to do the following three things every semester:

\begin{itemize}
    \item Pay the semester fee once you get notified via mail, to confirm that you want to keep studying. You can look up the details of due and completed payments in Alma > My Studies > Student Service in the tab \quotes{Payments}.
    
    \item Prove to the \quotes{Studierendenwerk} that you are still studying and therefore entitled to keep your room by uploading your \quotes{Immatrikulationsbescheinigung} for the next semester \href{https://tl1.eu/SWTUE/#maintenance/upload}{$\rightarrow{}$ here}. You can find the PDF in Alma > My Studies > Student Service in the tab \quotes{Requested Reports / Reports}.

    \item Revalidate your student-ID card, to be admitted to exams, use it as a payment option at university services like the mensa and our washing machine or get discounts in stores. For details have a look at section \ref{sec:qna-top-up-card}.
\end{itemize}

\section{That's a lot, is it all set and done now?}

Well, you should definetely check for your specific case what you need to do. There is help for foreign students, talk to us or other students you know or research on your own. But I hope we were able to give you advice on the most important first steps.

And don't forget to cancel and de-register everything before you leave! You wouldn't want to pay unnecessary fees, right?