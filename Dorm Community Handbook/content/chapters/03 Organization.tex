\chapter{How we organize ourselves} \label{chap:organization}
Now that you got to know us and how life is like around here, lets have a look on how we organize ourselves.

\section{Keeping the kitchen clean}
Ah yes, the kitchen. Keeping order with 15 students sharing one room is a challenge for sure...

But rest assured, that we do our best to keep the chaos at bay. The main objective of our organization has to be to keep the kitchen clean continuously, and not rely on frequent cleanups, because I can assure you: There is no way to hold a cleanup schedule that would keep up with the paste the kitchen will drown in misery ;).

That's why the first rule for the kitchen is: Ignore the official cleaning schedule, it's nonsense. Or would you like to clean up half a years worth of built up mess that noone cared about before and schedule your work and freetime activities around cleaning duty? Told you it was nonsense.

Instead, we ask you to take one simple rule to your heart: \textbf{Please clean up the mess you made yourself, and make sure to leave the kitchen \textit{at least} as clean as you found it.}

I want to emphasize the \quotes{at least} part in the above statement a little more, because the kitchen has gone through some unhealthy phases under the wrong mindset. Imagine the following situation: You go to the kitchen and find the stove a little dirty, so obviously someone before you did a bad job at cleaning up. If you just take care of it before or after you used the stove yourself, the situation is cleared within five minutes. But if you use the stove yourself and then refuse to clean up afterwards, because \quotes{you don't want to do someone elses job}, you are not just leaving the problem to someone else, you made it worse. Because no matter how careful you are, there are probably some sprinkles that snuck out the pot. And even if not, the heat will have lead to the previous mess drying and being harder to clean. The result is, that the next person would have to commit even more into cleaning up to clear the situation. That's why this mindset spreads increadibly quickly and in the end everyone is angry at each other and unhappy with the situation.

So please, be foregiving with your roomneighbors, and I'm sure they will return the favor and have your back when you had a long day and missed a spot yourself. And if you feel unhappy about something, please communicate and don't build up anger. We all use the kitchen differently and different cultural backgrounds can also lead to very different priorities. But so far everyone I have met in this community was willing to work towards betterment once an issue has been addressed.

Some final words to conclude this section: Even if there are some inhabitants that don't care about the kitchen at all, as long as a majority is working together the situation is manageable. So let's join forces, shall we?

\subsection{Kitchen responsibilities}

Unfortunately the kitchen does not only consist of flat surfaces that can be wiped clean after every use and the job is done. There are some bigger jobs, that need to be taken care of once in a while. Frequency and effort is very dependent on the kind of job and the how much the object in question is used.

To distribute the workload as fair as possible, we came up with a responsibility system. Let's go through the basics of this system using the oven as an example:
\begin{enumerate}
    \item Everyone who uses the oven, is still obligated to clean up the immediate mess they have made. For example, that includes to rinse the oven tray and to get rid of bigger splashes from boiling food or drips on the bottom.
    \item If you don't clean up after yourself properly, the responsible person is not only allowed, but even encuraged to contact you and demand, that you take care of that issue.
    \item In turn, the responsible person will take care of a more thorough cleaning of the oven once in a while, e.g. getting rid of small splashes and the general layer that will build up over time on all oven surfaces from the food vapours.
    \item If someone finds the oven in an inacceptable state, they can turn to the responsible person listed below.
    \item If you leave for longer periods of time (let's say, more than 4 days?) you should organize a stand-in that will take care of your responsibilities until you are back. Of course you shouldn't leave the big tasks to them ;)
\end{enumerate}

If you read this the first time, you might be thinking \quotes{This just sounds like a blaming system!}, and I agree, that wrongly executed this could lead to firing blames at each other. But the intention is quite the contrary. \quotes{Keeping the kitchen clean} can be a very daunting task, because there is a lot to do in total, so where do you start? The list below is supposed to show you that you are not alone in this effort, and to break down this big task into small, managable chunks that we can distribute and limit the workload of single people. I hope you can see the vision, and maybe take one responsibility for yourself!

You can find the current responsibility distribution in table \ref{tab:kitchen-responsibilities}.

\begin{table}[htp]
    \centering
    \begin{tabular}{ccc}
        \rowcolor[HTML]{F89646} 
        Responsibility            & Person      & Room Number \\ \hline
        Couch \& Speaker Area     & Robin Epple & U 13        \\ \hline
        Dishes                    &             &             \\ \hline
        Floor \& Shelves          & Robin Epple & U 13        \\ \hline
        Freezer                   &             &             \\ \hline
        Fridge                    &             &             \\ \hline
        Microwave                 &             &             \\ \hline
        Oven                      &             &             \\ \hline
        Table \& Cooking surfaces &             &             \\ \hline
        Toaster \& Kettle         &             &             \\ \hline
    \end{tabular}
    \caption{The current kitchen-responsibility distribution.}
    \label{tab:kitchen-responsibilities}
\end{table}

\section{Keeping track of finances}
I hate to break it to you, but living costs money... but wouldn't it be awful to shift around cents everey day for groceries, a trip to the cinema, etc.?

Well, at least we felt like it was, so we decided to do things a little different and open a group in the app \quotes{Splitwise}. If someone pays the expenses for multiple people, we don't immediately pay each other out, but instead enter the expenses and who owes how much in the app.

Ideally the expenses level out over time. But even if they don't we at least can pay out debts in less frequent intervals.

If you want to participate in our cooking or other activities, we kindly ask you to download the app from the \mbox{\href{https://play.google.com/store/apps/details?id=com.Splitwise.SplitwiseMobile&hl=de&pli=1}{$\xrightarrow{}$ Play Store (Android)}} or \mbox{\href{https://apps.apple.com/de/app/splitwise/id458023433}{$\xrightarrow{}$ App Store (iOS)}}. You can find the invitation link to our group as a QR code in the kitchen.

\section{Where your groceries go}