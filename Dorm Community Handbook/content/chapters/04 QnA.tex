\chapter{How to do ... ?} \label{chap:qna}
If you're new here and don't have any questions on how some things work in the WHO, you'd be the first. In fact, even most of our longtime residents stumble into questions once in a while.

In the following we try to answer some of the common questions in a QnA form. But don't worry if your question is not answered below, just ask us in chat and we will try to help!

\section{Pick up your mail}
In the center of the WHO area, directly next to the entrance to the club \quotes{Kuckuck} you will find the mail room. Just go through the rows and find the mailbox labelled with your address. The key for the mailbox should be part of the bundle you received at the beginning. 

Unfortunately there is no indication for when you got mail, so make sure to take a look regularily so you don't miss important letters.

\section{Receive parcels}
If you ordered a parcel, it is a little inconsistent where you can find it. But these are the four main spots:
\begin{description}
    \item[Personal handover:] For bigger or more valuable parcels, the delivery service will probably ring at your doorbell and hand them to you in person.
    \item[Mail room:] Smaller or less valuable parcels might get deposited in the mail room, in front of your mailbox.
    \item[Entrance area:] Another spot to look for the parcels is the main entrance to the house. They might deposit the parcels next to the doorbell-panel or inside the hallway.
    \item[Kitchen, floor or room door:] It's less likely, but the delivery service might bring the parcel down to our floor and deposit it in front of your door or in the kitchen.
\end{description}

\section{Top up your student-id card}
The student-id card is the main payment method for everything related to the university or the Studierendenwerk, e.g. the mensa and cafeteria, the washing machine or the printer. There are multiple ways to top up your card:
\begin{description}
    \item[Digital top-up stations:] In most university buildings you will find a station where you can top up your card, for example at the Morgenstelle there is one next to the staircase to the mensa. However, most of these only accept digital payment by card and are quite picky on what cards they accept. Give it a try with yours, since this is the most convenient way! But don't be afraid if your card is rejected, there are other methods.
    \item[Cash register in the Mensa:] If you eat at one of the university offerings (mensa, cafeteria), at least some cash registers allow payment by bank card. There you can also top up your student-id card, and the cash registers accept more cards than the digital top-up stations.
    \item[Cash top-up stations:] As far as I know there is only one top-up station that accepts physical cash. You can find it in the main university library \href{https://maps.app.goo.gl/ifwoMYV6JYGK6dXs7}{($\xrightarrow{}$ here)} in the city center, to the back in the ground floor area.
\end{description}

\section{Revalidate your student-id card}
Since it is possible to end your studies after any semester, your student-id card is only ever valid for the current semester. So when the new semester starts, you need to re-validate the card before you use it as payment method again.

To do that, you need to head to the main university library \href{https://maps.app.goo.gl/ifwoMYV6JYGK6dXs7}{($\xrightarrow{}$ here)} in the city center. On the ground floor head straight through to the back area. After you passed the first wall, head right. There you will find some automatic stations that will revalidate the card for you. You can check that the process was successful, by looking at the blue-ish printed text on the front of your card: It should have updated to the next semester.

\section{Use the washing machine}
The washing room of Fichtenweg 17 U is right on our floor. Head to the bottom of the main staircase and then (coming from the staircase) there is a metal door to the left.

The washing machine and dryer are shared between all inhabitants of the entire house, not just our floor. To use it, follow this procedure:
\begin{enumerate}
    \item Check that the washing machine is empty and not running a program, then put in your laundry.
    \item Afterwards you can enter the program you want and put in washing detergent. Important: To the right there is a shelf where a lot of washing detergent is standing around. These are not public, please get your own or ask one of us to share and don't steal from others. Of course you can deposit your own in that shelf as well.
    \item Next to the shelf there is a little box on the wall. Enter your student-id card, select either the washing machine or the dryer and pay for the program you selected.
    \item Now you can start the washing process. Don't worry, the machine is locked during the process, so noone can interrupt your program or even steal from your laundry, so you can go and do something else in the meantime. Just make sure to come back in time and get your laundry once the program finished.
    \item If you leave your laundry in one of the machines after the program has finished, there is a common agreement that someone needing the machine is allowed to pull it out and put it next to the machine, to start their own laundry.
    \item For drying, please use the dryer or hang your clothes in the back part of the laundry room. The Studierendenwerk doesn't want you to dry it in your room, because our rooms are not resistent to high levels of humidity and might start to grow mold.
\end{enumerate}

Using the dryer follows pretty much the same procedure: Put in your laundry, select a program, pay for the program, start the machine and come back in time.

\subsection{Known issues and solutions}
As with every public technology, it only works most of the time. Fortunately for you, there are lots of people around you that have been using these machines for a while and probably already have experience with the issue you are facing. In the following section we will add known issues and how they can be resolved, when we encounter a new one:
\begin{description}
    \item[Stuck in previous program:] It sometimes happens, that you find the washing machine unlocked, open and empty, but it wont let you select a program, because it is stuck in the end-phase of the previous program. The issue usually resolves, if you just rotate the barell manually for a few turns. The display should reset to the program selection.
\end{description}

\section{Use the printer}
In the back of the mail room, there is a public printer for all WHO residents. It will only do A4 paper format, yellow-ish recycling paper and black-and-white printing, but for most use cases that is enough.

To use it, do as follows:
\begin{enumerate}
    \item Export the file you want to print as PDF and save it to an USB device.
    \item On the left side of the printer there is an extension sticking out. Enter your student card there. PS: If you find another persons student card in there, please bring it to the \quotes{Wohnheimverwaltung}. You can find it in the same building, the entrance is exactly on the other side, towards the parking garage.
    \item Now look at the right side of the printer. You will find a USB Slot, where you can enter your device.
    \item Finally, look at the screen of the printer. Select the option to print from USB and navigate through your folders to the desired file.
    \item After selecting the file, there are some options displayed, but don't bother looking for too long, there is not much to choose from. The only relevant options are one- or two-sided print and the number of copies.
    \item Once you are happy with your settings press the big green button below the screen to start printing.
    \item After the print is complete, eject your USB device before pulling it out, to make sure no files get corrupted!
    \item \textbf{Please don't forget your student card!} To eject it from the slot press the corresponding button on the extension.
\end{enumerate}

\section{Set up your internet}
Our rooms are part of the university network. That means, you don't need an internet contract or do a lot of setup work. Just buy a router that allows connecting it to a LAN Port. Pretty much every router is able to do that, though it might be labelled as \quotes{WAN} instead of \quotes{LAN}. If you are interested in the difference, ask one of the techies on the floor ;). 

You don't need DSL or fiber optics compatibility, so you might want to save your money on that. If you don't have any idea on what to get, go to a local tech store (e.g. the Media Markt \href{https://maps.app.goo.gl/XmRXQEo6ihkXH9jS9}{$\xrightarrow{}$ here}) and ask for a router. The most common router company for home use in Germany is \quotes{FRITZ!Box} by AVM.

Setting up your WiFi is dependent on the router you bought. Follow the instruction manual or just click around the user interface, routers for home use are usually pretty self-explanatory and have sensible default settings.

If you want or need help: Again, just ask one of the techies on the floor, I'm sure someone will be happy to assist you.

\section{Find a parking spot}
Parking in Tübingen in general is not an easy task. If you own a vehicle and need a permanent parking spot, contact the \quotes{Wohnheimverwaltung}, they manage the parking garage. As of 2025, the monthly fee for a spot in that garage is 20€.

If you only need a temporary spot for moving in or having a visitor, there are multiple options. The Fichtenweg street offers quite a few parking lots on the side. They are all subject to a charge though, so make sure to get a ticket from the vending machine, which you can find \href{https://maps.app.goo.gl/Wi5gXiBErb9xiCWt9}{$\xrightarrow{}$ around here}.

Depending on what you want to do, another option might be to use the public parking lot in the shopping area across the street, next to the Edeka and the public pool. Mind the time and usage restrictions though!