\chapter{How to do ... ?} \label{chap:qna}
If you're new here and don't have any questions on how some things work in the WHO, you'd be the first. In fact, even most of our longtime residents stumble into questions once in a while.

In the following, we try to answer some of the common questions in a Q\&A form. But don't worry if your question is not answered below; just ask us in chat, and we will try to help!

\section{Pick up your mail} \label{sec:qna-mail}
You will find the mail room in the center of the WHO area, directly next to the entrance to the club \quotes{Kuckuck}. Just go through the rows and find the mailbox labeled with your address. The key for the mailbox should be part of the bundle you received at the beginning. 

Unfortunately, there is no indication of when you received mail, so make sure to check regularly so you don't miss important letters.

\section{Receive parcels}
If you ordered a parcel, it is a little inconsistent where you can find it. But these are the four main spots:
\begin{description}
    \item[Personal handover:] For bigger or more valuable parcels, the delivery service will probably ring your doorbell and hand them to you in person.
    \item[Mail room:] Smaller or less valuable parcels might get deposited in the mail room in front of your mailbox.
    \item[Entrance area:] Another spot to look for the parcels is the main entrance to the house. They might deposit the parcels next to the doorbell panel or inside the hallway.
    \item[Kitchen, floor or room door:] It's less likely, but the delivery service might bring the parcel down to our floor and deposit it in front of your door or in the kitchen.
\end{description}

If you are not home when a parcel is delivered, you can, of course, always ask one of us to accept the delivery, take it from the mail room/entrance area to your room door, or even store it until you are back.

\section{Top up your student-id card} \label{sec:qna-top-up-card}
The student-id card is the main payment method for everything related to the university or the Studierendenwerk, e.g., the mensa and cafeteria, the washing machine, or the printer. There are multiple ways to top up your card:
\begin{description}
    \item[Digital top-up stations:] In most university buildings, you will find a station to top up your card; for example, at the Morgenstelle, there is one next to the staircase to the mensa. However, most of these only accept digital payment by card and are pretty picky about what they take. Give it a try with yours since this is the most convenient way! But don't be afraid if your card is rejected; there are other methods.
    \item[Cash register in the Mensa:] If you eat at one of the university offerings (mensa, cafeteria), at least some cash registers allow payment by bank card. You can also top up your student ID card there, and the cash registers accept more cards than the digital top-up stations.
    \item[Cash top-up stations:] As far as I know, only one top-up station accepts physical cash. You can find it in the main university library \href{https://maps.app.goo.gl/ifwoMYV6JYGK6dXs7}{($\xrightarrow{}$ here)} in the city center, to the back in the ground floor area.
\end{description}

\section{Revalidate your student-id card}
Since it is possible to end your studies after any semester, your student-id card is only ever valid for the current semester. So when the new semester starts, you need to re-validate the card before you use it as a payment method again.

To do that, you need to head to the main university library \href{https://maps.app.goo.gl/ifwoMYV6JYGK6dXs7}{($\xrightarrow{}$ here)} in the city center. On the ground floor, head straight through to the back area. After you pass the first wall, head right. There, you will find some automatic stations to revalidate the card. You can check that the process was successful by looking at the blue-ish printed text on the front of your card: It should have been updated for the next semester.

\section{Use the washing machine} \label{sec:qna-washing-machine}
The washing room of Fichtenweg 17 U is right on our floor. Head to the bottom of the main staircase, and then (coming from the stairs), there is a metal door to the left.

The washing machine and dryer are shared between all inhabitants of the entire house, not just our floor. To use it, follow this procedure:
\begin{enumerate}
    \item Check that the washing machine is not in the middle of an ongoing program, then press the "open door" button and put it in your laundry.
    
    \item Afterwards, you can enter the program you want and put in washing detergent. Important: To the right is a shelf where a lot of washing detergent is standing around. These are not public; please get your own or ask one of us to share, and don't steal from others. Of course, you can deposit your own on that shelf as well. 
    
    \item For those new to German washing machines, the settings may look complicated but relatively simple. The options near the top-right of the wheel (like 40-60 in the "Buntwäsche" section) will suffice for most batches of general clothes. You may want to press the "Feinwäsche" option for more delicate fabric. Remember to separate your loads into light and dark loads.
    
    \item Next to the shelf, there is a little metal box on the wall. Enter your student ID card, and it will briefly display the remaining amount of money in your student card. Select either the washing machine or the dryer (in our building, option 1 corresponds to the washing machine and 2 to the dryer) and pay for the program you selected. Every washing machine run costs 1.50, and every dryer runs for one euro.
    
    \item Now, you can start washing by pressing the "start" button. Don't worry. The machine is locked during the process, so feel free to go and do something else. Just come back in time and get your laundry once the program finishes.
    
    \item If you leave your laundry in one of the machines after the program has finished, there is a standard agreement that someone needing the machine can pull it out and put it next to or on top of the machine to start their laundry.

    \item Open the washing machine door, remove the laundry, and close the door again. Be sure to turn the knob back to the "off/end" setting so the washing machine does not stay stuck in "Knitterschutz."    
    
    \item For drying, please use the dryer or hang your clothes in the back part of the laundry room. The Studierendenwerk doesn't want you to dry it in your room because our rooms are not resistant to high humidity levels and might start to grow mold.
\end{enumerate}

Using the dryer follows the same procedure: Put in your laundry, select a program (similarly, the top right options of the weel will work fine for most cases), pay for the program, start the machine, and return in time.

\subsection{Known issues and solutions}
As with every public technology, it only works most of the time. Fortunately for you, many people around you have been using these machines for a while and probably already have experience with the issue you are facing. In the following section, we will add known issues and how they can be resolved when we encounter a new one:
\begin{description}
    \item[Stuck in the previous program:] Sometimes, you find the washing machine unlocked, open, and empty, but it won't let you select a program because it is stuck in the end-phase of the previous program, "Knitterschutz." The issue usually resolves if you just rotate the barrel manually for a few turns and turn the knob of the washing machine off the previously selected option. You can then choose your desired option and proceed as explained above.
\end{description}

\section{Use the printer}
In the back of the main room is a public printer for all WHO residents. It can only print on A4 paper format, yellow-ish recycling paper, and black-and-white, but for most use cases, that is enough.

To use it, do as follows:
\begin{enumerate}
    \item Export the file you want to print as a PDF and save it to a USB device.
    \item On the left side of the printer, there is an extension sticking out. Enter your student card there. PS: If you find another person's student card in there, please bring it to the \quotes{Wohnheimverwaltung}. You can find it in the same building, the entrance is precisely on the other side, towards the parking garage.
    \item Now look at the right side of the printer. You will find a USB Slot where you can enter your device.
    \item Finally, look at the screen of the printer. Select the option to print from USB and navigate through your folders to the desired file.
    \item After selecting the file, some options are displayed, but don't bother looking for too long; there is not much to choose from. The only relevant options are one- or two-sided print and the number of copies.
    \item Once you are happy with your settings, press the big green button below the screen to start printing.
    \item After the print is complete, eject your USB device before pulling it out to ensure no files get corrupted!
    \item \textbf{Please don't forget your student card!} To eject it from the slot, press the corresponding button on the extension.
\end{enumerate}

\section{Set up your internet}
Our rooms are part of the university network. That means you don't need an internet contract or do much setup work. Just buy a router to connect it to a LAN port. Almost every router can do that, though it might be labeled as \quotes{WAN} instead of \quotes{LAN}. If you are interested in the difference, ask one of the techies on the floor ;). 

You don't need DSL or fiber optics compatibility, so you might want to save money. If you don't have any idea on what to get, go to a local tech store (e.g. the Media Markt \href{https://maps.app.goo.gl/XmRXQEo6ihkXH9jS9}{$\xrightarrow{}$ here}) and ask for a router. The most common router company for home use in Germany is \quotes{FRITZ!Box} by AVM.

Setting up your WiFi depends on the router you bought. Follow the instruction manual or just click around the user interface. Routers for home use are usually pretty self-explanatory and have sensible default settings.

If you want or need help again, just ask one of the techies on the floor; I'm sure someone will be happy to assist you.

\section{Find a parking spot}
Parking in Tübingen, in general, is not an easy task. If you own a vehicle and need a permanent parking spot, contact the \quotes{Wohnheimverwaltung}; they manage the parking garage. As of 2025, the monthly fee for a place in that garage is 20€.

There are multiple options if you only need a temporary spot to move in or have a visitor. The Fichtenweg street offers quite a few parking lots on the side. They are all subject to a charge, though, so make sure to get a ticket from the vending machine, which you can find \href{https://maps.app.goo.gl/Wi5gXiBErb9xiCWt9}{$\xrightarrow{}$ around here}.

Depending on your plans, you might also want to use the public parking lot in the shopping area across the street, next to the Edeka and the public pool. However, be aware of the time and usage restrictions!