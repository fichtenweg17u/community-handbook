\chapter{The dorm-life} \label{chap:activities}

In this chapter we want to give you a quick overview on what we do as a community and how you can participate!

\section{It's cooking time!} \label{sec:cooking}
Let's start with our most important way of staying in contact: Chatting around dinner time! You will regularily find people announcing in chat, what they will be cooking and when. For most evenings this is more tailored towards small groups of three or four, but there are also bigger events once in a while where eight to nine people come together. These messages are meant as an open invitation, and everyone is welcome to join in. Whether you want to help with the preparations, just come by for eating or invite to a dinner yourself: Welcome, to our little cooking group!

Due to different schedules, preferences and wake rythms the size and constellation of people changes daily, and sometimes we also just stumble into each other while independently preparing our meals. But that just makes the conversations more interesting ;). 

If you have intolerances, ethical restrictions or just certain preferences on food, please don't let that keep you from engaging with us! Though it might not be possible to respect all wishes at all times, we will certainly do our best to include everyone.  

\section{Do you hear the music?} \label{sec:speakers}
The signature landmark in our kitchen is closely related to the cooking: Our speaker system in the back. Though they were purchased individually by me\footnote{Hi, Robin here writing this, look at section \ref{sec:robinE} if you want to get to know me better :)} they are meant to be free to use for everyone. We regularily play music while cooking or just hanging out in the kitchen. And the best part: Everyone brings different music from their country or individual interest! My playlist has already gotten quite a bit more colorful, I am excited to see what I can steal from yours ;)

If you want to use the speakers, I just ask you for two simple things:
\begin{enumerate}
    \item Treat the speakers with the respect they deserve. What you are looking at is a $\sim$600€ system, so please take care of it. You don't have to handle it with kid gloves, just some common sense: Don't throw heavy objects in the direction of the speakers, don't place drinks on top or next to them, etc.
    \item Treat your neighbors with the respect they deserve. Everyone at every time has a veto right to demand a lower volume or even the music to be turned off. Our walls are actually pretty good sound isolators, so this is usually not a problem. However, if noise is spilling to your room and keeping you from concentrating or sleeping, that is a pretty horrible feeling. So please make use of that right, and also respect the needs of others!
\end{enumerate}

As a general rule of thumb to not disturb anyone: \textbf{Keep the door closed} while you are in the kitchen, that makes a world of a difference. We have also agreed on some loose Night-Rest-rules: Try to keep it quiet \textbf{between 10pm and 6am before workdays\footnote{Workdays are meant as Monday to Friday, so quiet down on So, Mo, Tu, We, Th evening.}} and \textbf{between 12pm and 8am on the weekend\footnote{Fr, Sa}}. These \quotes{rules} are more meant as a loose guideline for everyone to have a similar understanding of the \quotes{default behaviour}, than to restrict you in any way or form. If you plan on longer evenings for special (or also not so special) events, just talk to us, and we will find a solution, I'm sure!

Lastly some technical guidance: The grey box in the middle is a smart receiver, so if you are connected to the WiFi (see section \ref{sec:wifi}) it should just show up as an AirPlay, Chromecast Audio, Spotify Connect, Tidal Connect, ... target. It is named \textit{Fichtenweg 17 U Kitchen} so I'm sure you'll find it. \textbf{Currently we are still missing a WiFi repeater in the kitchen, so the receiver is sadly out of reach. The repeater will be installed shortly, but for now contact me (Robin, see \ref{sec:robinE}) for adding it as a bluetooth device, it is a little stubborn in that regard...}

\section{Community WiFi} \label{sec:wifi}
Sadly, our kitchen is not equipped with LAN ports or a public WiFi router maintained by the Studierendenwerk. Fortunately there are enough tech-savvy people around to find a solution :D.

Currently the network \textit{Fichtenweg 17 U Kitchen} is hosted from room U 13 since that is the closest room to the kitchen. It is then bridged over using a WiFi repeater (\textbf{Coming soon...}). So if something is not working properly, please contact the responsible person as stated in chapter \ref{chap:responsibilities} and not the Studierendenwerk.

If you made it to this document I imagine you have already found it, but just to be sure: To connect to the WiFi just scan the QR-code in the kitchen.

\section{Everyday needs}

\section{Cool places around}