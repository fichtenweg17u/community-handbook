\chapter{The dorm-life} \label{chap:activities}

In this chapter, we want to give you a quick overview of what we do as a community and how you can participate!

\section{It's cooking time!} \label{sec:cooking}
Let's start with our most important way of staying in contact: Chatting around dinner time! You will regularly find people announcing what and when they will be cooking in the chat. For most evenings, this is more tailored towards small groups of three or four, but there are also bigger events once in a while where eight to nine people come together. These messages are meant as an open invitation; everyone is welcome to join in. Whether you want to help with the preparations, just come by to eat or invite us to dinner yourself: Welcome to our little cooking group!

Due to different schedules, preferences, and wake rhythms, the size and constellation of people change daily. Sometimes, we also just stumble into each other while independently preparing our meals. But that just makes the conversations more interesting. 

If you have intolerances, ethical restrictions, or just certain food preferences, please don't let that keep you from engaging with us! Though it might not be possible to respect all wishes at all times, we will certainly do our best to include everyone.

\section{Do you hear the music?} \label{sec:speakers}
The signature landmark in our kitchen is closely related to the cooking: Our speaker system in the back. Though they were purchased individually by me\footnote{Hi, Robin here writing this, look at section \ref{sec:robinE} if you want to get to know me better :)} they are meant to be free to use for everyone. We regularily play music while cooking or just hanging out in the kitchen. And the best part: Everyone brings different music from their country or individual interest! My playlist has already gotten quite a bit more colorful, I am excited to see what I can steal from yours ;)

If you want to use the speakers, I just ask you for two simple things:
\begin{enumerate}
    \item Treat the speakers with the respect they deserve. You are looking at a $\sim$900€ system, so please take care of it. You don't have to handle it with kid gloves; it's just some common sense: Don't throw heavy objects in the direction of the speakers, don't place drinks on top or next to them, etc.
    \item Treat your neighbors with the respect they deserve. Everyone at every time has a veto right to demand a lower volume or even the music to be turned off. Our walls are good sound isolators, so this is usually not a problem. However, if noise spills into your room and keeps you from concentrating or sleeping, that is a horrible feeling. So please use that right, and also respect the needs of others!
\end{enumerate}

As a general rule of thumb to not disturb anyone: \textbf{Keep the door closed} while you are in the kitchen makes a world of difference, we have also agreed on some loose Night-Rest-rules: Try to keep it quiet \textbf{between 10 pm and 6 am before workdays\footnote{Workdays are meant as Monday to Friday, so quiet down on So, Mo, Tu, We, Th evening.}} and \textbf{between 12 pm and 8 am on the weekend\footnote{Fr, Sa}}. These \quotes{rules} are more meant as a loose guideline for everyone to have a similar understanding of the \quotes{default behavior} than to restrict you in any way or form. If you plan on longer evenings for special (or not-so-special) events, just talk to us, and we will find a solution, I'm sure!

Lastly, some technical guidance: Somewhere hidden beneath the shelf is a smart receiver, so if you are connected to the WiFi (see section \ref{sec:wifi}), it should just show up as an AirPlay, Chromecast Audio, Spotify Connect, Tidal Connect, ... target. It is named \textit{Fichtenweg 17 U Kitchen} so I'm sure you'll find it.

\section{Attention: Movie nerds nearby!}

If you have been to the kitchen, you might have struggled to miss the big TV in the back. This is the newest addition to our little community and pretty much follows the same rules as with the speakers: Take care of the equipment and respect your neighbors, but don't forget to have fun using it! I would like to introduce two additional rules though:

\begin{enumerate}
    \item Please do never log into any accounts on the TV itself. You should respect that out of your own interest for the sake of protecting your privacy, but we especially don’t want to deal with accounts openly accessible that have payment information connected. Please just bring your own device and connect it to the HDMI cable that is available beneath the TV. It is 3m long, so just pull carefully if you need more slack and hide it again behind the shelf once you're done. The TV is only connected to the WiFi for screen casting and nothing else!

    \item Please announce in chat if you want to use the TV, just so your neighbors know and we don’t stumble into a situation where multiple people planned to use it at the same time. It might also give other people the chance to join in or raise a veto because they need to study or sleep early.
\end{enumerate}

\section{Community WiFi} \label{sec:wifi}
Sadly, our kitchen is not equipped with LAN ports or a public WiFi router maintained by the Studierendenwerk. Fortunately, there are enough tech-savvy people around to find a solution :D.

Currently, the network \textit{Fichtenweg 17 U Kitchen} is hosted from room U13 since that is the closest room to the kitchen. It is then bridged over using a WiFi repeater. So if something is not working correctly, please contact the responsible person as stated in chapter \ref{chap:responsibilities} and not the Studierendenwerk.

If you made it to this document, I imagine you have already found it, but just to be sure: to connect to the WiFi, scan the QR code in the kitchen.

\section{Everyday needs}
Of course, while living in Tübingen, you'll need to cover the basic needs, from food and hygiene to medical treatment. Luckily, the WHO is in a very good spot, and you won't need to travel far for either of these.  

\subsection{Groceries \& more}
For groceries, there are many different options in Tübingen, and depending on your daily schedule and food preferences, it is probably best if you look at the map yourself. I just want to list a few standard places we tend to go to:

\begin{description}
    \item[Edeka]
    The closest grocery shop is the Edeka right next to the WHO \href{https://maps.app.goo.gl/DTd9KutteiG8m5L1A}{($\xrightarrow{}$ here)}. It is not the cheapest option, but it offers a wide range of groceries. It is split into two parts: One sells groceries, and in the other, you will find drinks, hygiene products, and basic household needs. It also contains a little Bakery.
    
    \item[Penny]
    You will find a little shopping area between the \quotes{Morgenstelle} and the WHO. Among the shops there is a Penny \href{https://maps.app.goo.gl/7F1s1fUtsefv49gv8}{($\xrightarrow{}$ here)}. This can be a convenient stop on the way, and it is also a little cheaper than the Edeka. However, it is a little smaller and offers much less household goods. A bakery can be found a few buildings further.

    \item[Rewe]
    Another \quotes{on the way} option is the Rewe, which is close to the lecture buildings in the city center \href{https://maps.app.goo.gl/Sj6d82QsJSTaGCuv5}{($\xrightarrow{}$ here)}. Regarding groceries, it also offers a large product range, and it also has a bakery. For households, it is also less equipped than the Edeka.

    \item[Bio / Regional / ...]
    If you care about the sources of your food and/or the environmental impact of the options above, the Edeka offers the widest range of \quotes{Bio} labeled products, and the Rewe offers the most \quotes{regional} labeled products. There is also a specialized Bio-shop uphill \href{https://maps.app.goo.gl/EEEsyTK1hGbJ5H1B6}{($\xrightarrow{}$ here)} and an \quotes{Alnatura} shop in the city center \href{https://maps.app.goo.gl/8ppTPAT6ZfP2TXeg6}{($\xrightarrow{}$ here)}.

    \item[Asian Food]
    {\color{darkred} TODO}
\end{description}

\subsection{Medical}
If you are looking for medicine or medical treatment, there are a few places to keep in mind:

\begin{description}
    \item[Pharmacy] \ \
        \begin{itemize}
            \item A little bit uphill, we have a pharmacy to get your prescriptions \href{https://maps.app.goo.gl/cpDLc9VZ7XukpgS18}{($\xrightarrow{}$ here)}.
            \item If you are downtown, you have one at the end of Wilhelmstraße \href{https://maps.app.goo.gl/YXwvVpp81eTSZeJW7}{($\xrightarrow{}$ here)}.
        \end{itemize}
  
    \item[Doctor]
        Unfortunately, this is a topic we haven’t had much contact with. Tübingen has some general practitioners, but getting an appointment with them or becoming one of their patients is challenging. Some contributing factors include the fact that many doctors do not accept new patients, and you must find one that partners with your health insurance. However, most doctors have a designated day when you can walk in for a consultation. The best tip for getting through this is to push yourself to not delay and start getting a doctor as soon as possible. The German medical system is full of bureaucracy and waiting times, and the only way through is to file the paperwork with guidance and, unfortunately, wait the needed time.   
        
    \item[Hospital]
        In case of more serious situations, you should pay a visit to our hospital \href{https://maps.app.goo.gl/EHyduztd2QgxTVer9}{($\xrightarrow{}$ here)}.

    \item[EMERGENCY]
    Remember that in case of emergencies, you should call \textbf{112}.
\end{description}

\subsection{Postal services}
\begin{description}
    \item[Mail room]
    In the center of the WHO living area, you will find a building containing the dormitory administration, the student club \quotes{Kuckuck}, and also a mail room \href{https://maps.app.goo.gl/W9UtqfhE1LJJ5oGH8}{($\xrightarrow{}$ here)}. Inside, you will find a small mailbox with your room number, where letters will be delivered. You should regularly check your mailbox since there is no indication of when you received mail.
    \item[Postal office]
    A little uphill in the small \quotes{Nordring} shopping center, there is also a postal office \href{https://maps.app.goo.gl/nMnEheHexNQ5NZY37}{($\xrightarrow{}$ here)}. There you can send parcels and find essential office products like different pens, markers and everything necessary to send your parcels like packaging tape.
\end{description}

\section{Cool places around}
Of course, your life in Tübingen should not only be about survival; you can rest assured there is enough to do. The following list is supposed to give you an idea and will hopefully be ever-growing. So, if you have an incredible experience you want to share with your room neighbors, we would love to add a section!

\subsection{Kuckuck}
The closest you can find someplace to go is the student club \quotes{Kuckuck} just next door. They offer different events with special music and are open most evenings. Next to the Kuckuck is a little area where you can hang out and play table tennis.

\subsection{Public pool}
One of the closest activities around is the public pool right across the street \href{https://maps.app.goo.gl/NZaCajzPTgFjabqH6}{($\xrightarrow{}$ here)}. It's nothing special, so don't expect a waterpark or many different pool categories, but if you just want to swim a few rounds, that is the place to go. It also offers a sauna. The prices are not necessarily cheap for what it offers, but also not too overpriced, so considering the proximity, it is a good choice. If you want to make swimming your primary sport, you might be interested in looking into a yearly ticket, but that is only worth it if you go swimming at least once a week for most of the year.

\subsection{Cinema(s)}
Tübingen has three cinema theaters that cooperate and have a shared program. They offer a lot of different movies in different languages and subtitle versions (OmdU = Original audio with German subtitles and Omeu = Original audio with English subtitles), so if you're interested, just look around their program \href{https://tuebinger-kinos.de/programmuebersicht/}{($\xrightarrow{}$ here)}.
If you are very much into movies, they also offer a subscription to visit the cinema as often as you like \href{https://tuebinger-kinos.de/unlimited/}. But of course, that is only worth it if you plan on visiting the cinema at least twice a month.
(For questions about the cinema in Tübingen, you can refer to Luke; he'll be very happy to assist :))

If you are looking for a special experience, you might be interested in the IMAX in Leonberg. At the time of writing, this movie theater boasts the largest screen in the world and offers truly special image quality, 3D, and exceptional sound. So, if you find a movie that can take advantage of that, you will have a great time. You can find their program here \href{https://imax.traumpalast.de/index.php/PID/11295.html}{($\xrightarrow{}$ here)}.

\subsection{Bowling}
There are multiple places you can go to for bowling or billiards. I personally have only been to \quotes{Riverside Bowling} \href{https://maps.app.goo.gl/dgMVABnYTJC7qGBT8}{($\xrightarrow{}$ here)}, and it was definitely good fun :).  

\subsection{Botanical Garden}
Right next to the Morgenstelle, there is a big botanical garden. I'm sure you'll come across it some time, but it might be worth it to go and take a closer look around. It is a beautiful place to walk and enjoy some fresh air. There is also a tropical house with a cool feature: From the back to the left, there is a staircase to a little viewing platform that lets you view the jungle landscape from above.