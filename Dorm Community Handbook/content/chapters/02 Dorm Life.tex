\chapter{The dorm-life} \label{chap:activities}

In this chapter we want to give you a quick overview on what we do as a community and how you can participate!

\section{It's cooking time!} \label{sec:cooking}
Let's start with our most important way of staying in contact: Chatting around dinner time! You will regularily find people announcing in chat, what they will be cooking and when. For most evenings this is more tailored towards small groups of three or four, but there are also bigger events once in a while where eight to nine people come together. These messages are meant as an open invitation, and everyone is welcome to join in. Whether you want to help with the preparations, just come by for eating or invite to a dinner yourself: Welcome, to our little cooking group!

Due to different schedules, preferences and wake rythms the size and constellation of people changes daily, and sometimes we also just stumble into each other while independently preparing our meals. But that just makes the conversations more interesting ;). 

If you have intolerances, ethical restrictions or just certain preferences on food, please don't let that keep you from engaging with us! Though it might not be possible to respect all wishes at all times, we will certainly do our best to include everyone.  

\section{Do you hear the music?} \label{sec:speakers}
The signature landmark in our kitchen is closely related to the cooking: Our speaker system in the back. Though they were purchased individually by me\footnote{Hi, Robin here writing this, look at section \ref{sec:robinE} if you want to get to know me better :)} they are meant to be free to use for everyone. We regularily play music while cooking or just hanging out in the kitchen. And the best part: Everyone brings different music from their country or individual interest! My playlist has already gotten quite a bit more colorful, I am excited to see what I can steal from yours ;)

If you want to use the speakers, I just ask you for two simple things:
\begin{enumerate}
    \item Treat the speakers with the respect they deserve. What you are looking at is a $\sim$600€ system, so please take care of it. You don't have to handle it with kid gloves, just some common sense: Don't throw heavy objects in the direction of the speakers, don't place drinks on top or next to them, etc.
    \item Treat your neighbors with the respect they deserve. Everyone at every time has a veto right to demand a lower volume or even the music to be turned off. Our walls are actually pretty good sound isolators, so this is usually not a problem. However, if noise is spilling to your room and keeping you from concentrating or sleeping, that is a pretty horrible feeling. So please make use of that right, and also respect the needs of others!
\end{enumerate}

As a general rule of thumb to not disturb anyone: \textbf{Keep the door closed} while you are in the kitchen, that makes a world of a difference. We have also agreed on some loose Night-Rest-rules: Try to keep it quiet \textbf{between 10pm and 6am before workdays\footnote{Workdays are meant as Monday to Friday, so quiet down on So, Mo, Tu, We, Th evening.}} and \textbf{between 12pm and 8am on the weekend\footnote{Fr, Sa}}. These \quotes{rules} are more meant as a loose guideline for everyone to have a similar understanding of the \quotes{default behaviour}, than to restrict you in any way or form. If you plan on longer evenings for special (or also not so special) events, just talk to us, and we will find a solution, I'm sure!

Lastly some technical guidance: The grey box in the middle is a smart receiver, so if you are connected to the WiFi (see section \ref{sec:wifi}) it should just show up as an AirPlay, Chromecast Audio, Spotify Connect, Tidal Connect, ... target. It is named \textit{Fichtenweg 17 U Kitchen} so I'm sure you'll find it. \textbf{Currently we are still missing a WiFi repeater in the kitchen, so the receiver is sadly out of reach. The repeater will be installed shortly, but for now contact me (Robin, see \ref{sec:robinE}) for adding it as a bluetooth device, it is a little stubborn in that regard...}

\section{Community WiFi} \label{sec:wifi}
Sadly, our kitchen is not equipped with LAN ports or a public WiFi router maintained by the Studierendenwerk. Fortunately there are enough tech-savvy people around to find a solution :D.

Currently the network \textit{Fichtenweg 17 U Kitchen} is hosted from room U 13 since that is the closest room to the kitchen. It is then bridged over using a WiFi repeater (\textbf{Coming soon...}). So if something is not working properly, please contact the responsible person as stated in chapter \ref{chap:responsibilities} and not the Studierendenwerk.

If you made it to this document I imagine you have already found it, but just to be sure: To connect to the WiFi just scan the QR-code in the kitchen.

\section{Everyday needs}
Of course while living in Tübingen you'll need to cover the basic needs, from food and hygene to medical treatment. Luckily the WHO is in a very good spot and you won't need to travel far for either of these.  

\subsection{Groceries \& more}
For groceries there are a lot of different options in Tübingen, and depending on your daily schedule and food preferences it is probably best if you look at the map yourself. I just want to list a few standard-places we tend to go to:

\begin{description}
    \item[Edeka]
    The closest groceries shop is the Edeka right next to the WHO \href{https://maps.app.goo.gl/DTd9KutteiG8m5L1A}{($\xrightarrow{}$ here)}. It is not the cheapest option, but has a wide range of groceries to offer. It is split in two parts: One sells the grocieres, and in the other one you will find drinks, hygene products and basic household needs. It also contains a little Bakery.

    \item[Penny]
    Between the \quotes{Morgenstelle} and the WHO you will find a little shopping area. Among the shops there is a Penny \href{https://maps.app.goo.gl/7F1s1fUtsefv49gv8}{($\xrightarrow{}$ here)}. This can be a convienient stop on the way and it is also a little cheaper than the Edeka. It is a little smaller however and offers much less household goods. A bakery can be found a few buildings further.

    \item[Rewe]
    Another \quotes{on the way} option is the Rewe, that is close to the lecture buildings in the city center \href{https://maps.app.goo.gl/Sj6d82QsJSTaGCuv5}{($\xrightarrow{}$ here)}. Regarding groceries it also offers a big product range, and it also contains a bakery. For household it is also less equipped than the Edeka.

    \item[Bio / Regional / ...]
    If you care about the sources of your food and / or environmental impact, of the options above the Edeka offers the widest range of \quotes{Bio} labelled products, and the Rewe the most \quotes{regional} labelled products. There is also a specialized Bio-shop uphill \href{https://maps.app.goo.gl/EEEsyTK1hGbJ5H1B6}{($\xrightarrow{}$ here)} and an \quotes{Alnatura} shop in the city center \href{https://maps.app.goo.gl/8ppTPAT6ZfP2TXeg6}{($\xrightarrow{}$ here)}.

    \item[Asian Food]
    {\color{darkred} TODO}
\end{description}

\subsection{Medical}
If you are looking for medicine or medical treatment, there are a few places to keep in mind:

\begin{description}
    \item[Pharmacy]
    A little bit uphill we have a pharmacy to get your prescriptions \href{https://maps.app.goo.gl/cpDLc9VZ7XukpgS18}{($\xrightarrow{}$ here)}.
    \item[Doctor]
    {\color{darkred} TODO}
    \item[Hospital]
    {\color{darkred} TODO}
\end{description}

\subsection{Postal services}
\begin{description}
    \item[Mail room]
    In the center of the WHO living area you will find a building containing the dormitory administration, the student club \quotes{Kuckuck} and also a mail room \href{https://maps.app.goo.gl/W9UtqfhE1LJJ5oGH8}{($\xrightarrow{}$ here)}. Inside you will find a small mailbox with your room number, that letters, etc. will be delivered to. You should regularily check your mailbox, since there is no other indication for when you received mail.

    \item[Postal office]
    A little uphill in the small \quotes{Nordring} shopping center there is also a postal office \href{https://maps.app.goo.gl/nMnEheHexNQ5NZY37}{($\xrightarrow{}$ here)}. There you can send parcels and find basic office products like different pens, markers and of course everything necessary to send your parcels like packaging tape.
\end{description}

\section{Cool places around}
Of course your life in Tübingen should not only be about survival, and you can rest assured, there is enough to do. The following list is supposed to give you an idea and will hopefully be evergrowing. So if you come across a cool experience that you want to share with your roomneighbours, we would love to add a section!

\subsection{Kuckuck}
The closest you can find some place to go is the student club \quotes{Kuckuck} just next door. They offer different events with special music and are opened most evenings. Next to the Kuckuck there is also a little area to hang out and play table tennis.

\subsection{Public pool}
One of the closest activities around is the public pool right across the street \href{https://maps.app.goo.gl/NZaCajzPTgFjabqH6}{($\xrightarrow{}$ here)}. It's nothing special, so don't expect a waterpark or many different pool categories, but if you just want to go swim a few rounds, that is the place to go. It also offers a sauna. The prices are not necessarily cheap for what it offers, but also not too overpriced, so considering the proximity it is a good choice. If you want to make swimming your main sports, you might be interested to look into a yearly ticket, but that is only worth it if you go swimming at least once a week for most of the year.

\subsection{Cinema(s)}
Tübingen has three cinemas, that cooperate and have a shared program. They offer a lot of different movies in different language and subtitle versions, so if you're interested just have a look around their program: \url{https://tuebinger-kinos.de/programmuebersicht/}.
If you are very much into movies, they also offer a subscription where you can visit the cinema as often as you like. But of course that is only worth it if you plan on visiting the cinema a lot \^{}\^{}.

If you are looking for a special experience, you might be interested in the IMAX in Leonberg. At the time of writing, this movie theater boasts the largest screen in the world and offers truly special image quality, 3D and exceptional sound. So if you find a movie that can take advantage of that, you will definetly have a great time. You can find their program here: \url{https://imax.traumpalast.de/index.php/PID/11295.html}.

\subsection{Bowling}
There are multiple places you can go to for bowling or billiard. I personally have only been to \quotes{Riverside Bowling} \href{https://maps.app.goo.gl/dgMVABnYTJC7qGBT8}{($\xrightarrow{}$ here)} and it was definetely good fun :). 

\subsection{Botanical Garden}
Right next to the Morgenstelle, there is a big botanical garden. I'm shure you'll come across it some time, but it might be worth it to go and take a closer look around. It is a beautiful place to walk around and enjoy some fresh air. There is also a tropical house with a cool feature: In the back to the left there is a staircase to a little viewing plattform, that allows you to have a view from above the trees on the jungle landscape